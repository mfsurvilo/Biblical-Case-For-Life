\chapter{Introduction}
\begin{fullwidth}
How should God's word shape our position on abortion?\\
Should believers be involved in politics?\\

In America there were an estimated 930,000 abortions in 2020. There were 3,605,201 live births in 2020. This means that 
that we aborted over 25\% of the children conceived in 2020.\\ Many will say that abortion is a political issue, but is it?\\  Others will say that abortion is a woman's right. 
God's word says that as believers we have the ability to discern right from wrong. Paul also tells us that we have the mind of Christ. This means that followers of Christ have no 
excuses for not holding an informed opinion on the sanctity of life.\\

In a world where the value of life is often debated, scripture invites us to look beyond societal opinions into the very heart of God. 
From the first chapters of Genesis through the psalms and prophets all the way through Revelation, the Bible reveals a Creator who intricately forms life;
knows us before we are born, and assigns purpose to every person. This study seeks to guide believers through God's Word, uncovering His deep love for the unborn and the 
responsibility we have as followers of Christ if we are to live with obedience.  By examining key passages, we will reflect on God's character 
and align our hearts with His in defending the most vulnerable. \\
\end{fullwidth}
\vspace{1\baselineskip}
\footnote{ADD REAL LINK TO STATISTICS> FIX FIRST PARAGRAPH - CLUNKY}

This study will operate under the a few assumptions.
\vspace{1\baselineskip}

\textbf{1. The Bible as the Word of God:} 
\marginnote{
    \begin{scripture}[2 Timothy 3:16]
            \vs{16} All Scripture is inspired by God and profitable for teaching, for reproof, for correction, for training in righteousness;
    \end{scripture}
}
We believe the Bible is God's inspired Word. Both the Old and New Testaments reveal His character, intentions, and heart for humanity, offering truths that are timeless and relevant.
\vspace{2\baselineskip}


\textbf{2. Parallels Between Ancient Israel and Followers of Christ:}
\marginnote{
    \begin{scripture}[1 Corinthians 10:11]
            \vs{11} Now these things happened to them as an example, and they were written for our instruction, upon whom the ends of the ages have come.
    \end{scripture}
}

We assume that God's relationship with ancient Israel offers valuable insights for today's believers, providing examples of His guidance, discipline, and faithfulness that still apply.\\
\vspace{2\baselineskip}

\textbf{3. Consistent Character of God:} 
\marginnote{
    \begin{scripture}[James 1:17]
            \vs{17} Every good thing given and every perfect gift is from above, coming down from the Father of lights, with whom there is no variation or shifting shadow.
    \end{scripture}
}
Although, through Christ, believers are not subject to the same Old Testament punishments for sin, we trust that God's values and character remain unchanged. The principles seen in the Old Testament still help us understand who God is and what He values.\\

\vspace{2\baselineskip}

Throughout this study verses will be shown in the NASB 2020 version. And the words of Christ will be noted with an asterisk.
\textbf{NOTE TO SELF FORMAT SCRIPTURE APPROPRIATELY}
