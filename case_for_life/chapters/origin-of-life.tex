\chapter{Origin of Life}
\section{The Womb}

\begin{scripture}[Luke 1:41-44]
    \ch{1} \marginnote{\question{In the following passage was the baby (John the Babptist) conscious of 
    his sorroundings?}}
    When Elizabeth heard Mary's greeting, the baby leaped in her womb; and Elizabeth was filled with the Holy Spirit.    
        \vs{42}And she cried out with a loud voice and said, 'Blessed [are] you among women, and blessed [is] the fruit of your womb!
        \vs{43}And how has it [happened] to me, that the mother of my Lord would come to me?
        \vs{44}For behold, when the sound of your greeting reached my ears, the baby leaped in my womb for joy.
        \marginnote{\question{In light of \textbf{2 Timothy 3:16} (see introduction) what might be God's purpose in recording this moment into the Gospel of Luke?}}
 \end{scripture}

\vspace{1\baselineskip}

\begin{scripture}[Psalm 139:13-16]

    \ch{139}    \marginnote{\question{Who controls has ultimate control over a woman's body?}}
        For You formed my inward parts; You wove me in my mother's womb.  
       \vs{14}I will give thanks to You, for I am fearfully and wonderfully made; Wonderful are Your works, And my soul knows it very well.  
        \vs{15}My frame was not hidden from You, When I was made in secret, [And] skillfully wrought in the depths of the earth;  
        \vs{16}Your eyes have seen my unformed substance; And in Your book were all written The days that were ordained [for me], When as yet there was not one of them.  
\end{scripture}

\vspace{1\baselineskip}

\begin{scripture}[Job 31:15]
    \marginnote{NEED TO ADD CONTEXT TO THIS>>>}
    \vs{15}Did not He who made me in the womb make him, And the same one fashion us in the womb?
\end{scripture}

\vspace{1\baselineskip}

\begin{scripture}[Genesis 29:31]
    \vs{31}Now the LORD saw that Leah was unloved, and He opened her womb, but Rachel was barren.
\end{scripture}

\vspace{1\baselineskip}

\begin{scripture}[Genesis 30:22]
    \vs{22}Then God remebered Rachel, and God gave heed to her and opened her womb.
\end{scripture}

\vspace{1\baselineskip}

\begin{scripture}[1 Samuel 1:5-6]
    \vs{5}but to Hannah he would give a double portion, for he loved Hannah, but the LORD had closed her womb.
    \vs{6}Her rival, however, would provoke her bitterly to irritate her, because the LORD had closed her womb.
\end{scripture}

\newpage
\section{Children as Fulfillment of God's Plan}

The following passage is about the birth of John the Baptist, who was to prepare the way for Jesus Christ.
\vspace{1\baselineskip}

\begin{scripture}[Luke 1:11-25]
    \ch{1}Then an angel of the Lord appeared to him, standing at the right side of the altar of incense.
    \vs{12}When Zechariah saw him, he was startled and was gripped with fear. 
    \vs{13}But the angel said to him: “Do not be afraid, Zechariah; your prayer has been heard. Your wife Elizabeth will bear you a son, and you are to call him John. 
    \vs{14}He will be a joy and delight to you, and many will rejoice because of his birth, 
    \vs{15}for he will be great in the sight of the Lord. He is never to take wine or other fermented drink, and he will be filled with the Holy Spirit even before he is born. 
    \vs{16}He will bring back many of the people of Israel to the Lord their God. 
    \vs{17}And he will go on before the Lord, in the spirit and power of Elijah, to turn the hearts of the parents to their children and the disobedient to the wisdom of the righteous—to make ready a people prepared for the Lord.”
    \vs{18}Zechariah asked the angel, “How can I be sure of this? I am an old man and my wife is well along in years.”
    \vs{19}The angel said to him, “I am Gabriel. I stand in the presence of God, and I have been sent to speak to you and to tell you this good news. 
    \vs{20}'And behold, you shall be silent and unable to speak until the day when these things take place, because you did not believe my words, which will be fulfilled in their proper time.'
    \vs{21}The people were waiting for Zacharias, and were wondering at his delay in the temple.
    \vs{22}But when he came out, he was unable to speak to them; and they realized that he had seen a vision in the temple; and he kept making signs to them, and remained mute.
    \vs{23}When the days of his priestly service were ended, he went back home.
    \vs{24}After these days Elizabeth his wife became pregnant, and she kept herself in seclusion for five months, saying,
    \vs{25}'This is the way the Lord has dealt with me in the days when He looked [with favor] upon [me], to take away my disgrace among men.'

\end{scripture}

\vspace{2\baselineskip}

\marginnote{\question{NOT SURE WHAT TO ASK ABOUT THIS ONE ?}}

\newpage

\section{God's Plan for Individuals}
\begin{scripture}[Isaiah 49:1]
    \vs{1}Listen to Me, O islands, And pay attention, you peoples from afar. The LORD called Me from the womb; From the body of My mother He named Me.
\end{scripture}

\vspace{4\baselineskip}

\begin{scripture}[Jeremiah 1:5]
    \vs{5}\marginnote{\question{God says He 'knew' and 'consecreated Jeremiah before he was born. What do you think it means to be 'known' by God in this way?}}
    'Before I formed you in the womb I knew you, And before you were born I consecrated you; I have appointed you a prophet to the nations.'
\end{scripture}

\vspace{4\baselineskip}

\begin{scripture}[Genesis 25:22-23]
    \vs{22}\marginnote{\question{What can this passage tell you about the potential God sees in every person?}}
    But the children struggled together within her; and she said, 'If it is so, why then am I [this way]?' So she went to inquire of the LORD.\\
    \vs{23}The LORD said to her, 'Two nations are in your womb; And two peoples will be separated from your body; And one people shall be stronger than the other; And the older shall serve the younger.'
\end{scripture}

\vspace{4\baselineskip}

\begin{scripture}[Galatians 1:15]
    \vs{15}But when God, who had set me apart [even] from my mother's womb and called me through His grace, was pleased
\end{scripture}

\vspace{4\baselineskip}

\marginnote{\question{Looking at all these verses together, what similarities do you see regarding God's involvement in people's lives before birth?}}

\vspace{10\baselineskip}
\marginnote{\question{How do these ideas impact your understanding of what it means to have purpose or calling in life? When is someone's purpose or calling determined?}}
