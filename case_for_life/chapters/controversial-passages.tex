\chapter{Controversial and Unique Passages}
\begin{fullwidth}
    
God`s character is unmoving and unchanging. As the Bible is a reflection of God`s character lived out in many different contexts, 
it behooves us to look at some of the more difficult to understand passages, and see how they fit into the broader narrative of God`s design. 
In this chapter, we will look at some of the passages that may be difficult to square with the rest of what we have been studying. 


\vspace{2\baselineskip}

Does God Sometimes Condone Child Sacrafice?

\vspace{1\baselineskip}

Did God Use Abortion as a Test of Infidelity?

\vspace{1\baselineskip}

Is Masturbation Morally Equivalent to Murder?

\end{fullwidth}





\pagebreak

\section{Does God Sometimes Condone Child Sacrafice?}
\begin{fullwidth}
    In the previous section we saw that God hates child sacrafice. However, in the story of Abraham and Isaac, God commands Abraham to sacrafice
    his son. This story is often used as an example of God`s cruelty, however, the story is not about child sacrafice, but about faith and obedience.
    To understand how Abraham could trust God`s plan, we have to look at their history together. God first comes to Abraham when he is was known as Abram and makes a convenant with him regarding his descendants.
\end{fullwidth}
\subsection{God Call`s Abraham}

\begin{scripture}[Genesis 12:1-7]
    \noindent\vs{1}Now the Lord said to Abram,\\
    “Go forth from your country,\\
    And from your relatives\\
    And from your father`s house,\\
    To the land which I will show you;\\
    \vs{2}And I will make you a great nation,\\
    And I will bless you,\\
    And make your name great;\\
    And so you shall be a blessing;\\
    \vs{3}And I will bless those who bless you,\\
    And the one who curses you I will curse.\\
    And in you all the families of the earth will be blessed.”\\

    \vspace{1\baselineskip}

    \vs{4}So Abram went forth as the \marginnote{\question{How old was Abram?}}
    Lord had spoken to him; and Lot went with him. Now Abram was seventy-five years old when he departed from Haran.
    \vs{5}Abram took Sarai his wife and Lot his nephew, and all their possessions which they had accumulated, and the persons which they had acquired in Haran, and they set out for the land of Canaan; thus they came to the land of Canaan.
    \vs{6}Abram passed through the land as far as the site of Shechem, to the oak of Moreh. Now the Canaanite was then in the land. \marginnote{\question{What is God`s promise to Abram?}}
    \vs{7}The Lord appeared to Abram and said, “To your descendants I will give this land.” So he built an altar there to the Lord who had appeared to him.
\end{scripture}

\pagebreak
\subsection{God Reaffirms His Covenant with Abram}

\begin{scripture}[Genesis 15:1-6]
\noindent\vs{1} \marginnote{Note: It is estimated that 8-10 years has passed since God first made his covenant with Abram in Genesis 12.}
    After these things  
    the word of the Lord came to Abram in a vision, saying,
    \vspace{1\baselineskip}
    \marginnote{\question{What is Abram`s concern?}}

    “Do not fear, Abram,\\ 
    I am a shield to you;\\
    Your reward shall be very great.”\\

    \vspace{1\baselineskip}

   \noindent\vs{2}Abram said, 
    “O Lord God, what will You give me, since I am childless, and the heir of my house is Eliezer of Damascus?” 
    \vs{3}And Abram said, “Since You have 
    given no offspring to me, one born in my house is my heir.” \marginnote{Note: Abram mentions Eliezer because it was typical that a master`s most trusted servant would inherit an estate if there were no biological heirs.}
    \vs{4}Then behold, the word of the Lord came to him, saying,
    “This man will not be your heir; but one who will come forth from your own body, he shall be your heir.”
    \vs{5}And He took him outside and said,\marginnote{\question{God reaffirms his promise to Abram. How does he describe what Abrams lineage will look like?}}
     “Now look toward the heavens, and count the stars, if you are able to count them.” 
     And He said to him, “So shall your descendants be.” 
    \vs{6}Then he believed in the Lord; and He reckoned it to him as righteousness.
\end{scripture}

\subsection{Isaac is the Heir, Not Ishmael}
\begin{fullwidth}
    In the time since Genesis 15, Abram and Sarai have still not had a child. They began to doubt God`s faithfullness and his ability to make Sarai`s womb fertile after all these years. In Genesis 16, Sarai suggests that Abram take her servant Hagar as a wife so that they may have a child. This was 
    a not an uncommon practice in ancient times. Hagar bears Abram a child and his name is Ishmael. Presumably, Abram and Sarai thought that Ishmael would be the child that God had promised them.
    \end{fullwidth}

    \vspace{1\baselineskip}

\begin{scripture}[Genesis 17:1-8]
    \vs{1}Now when Abram was ninety-nine years old, \marginnote{\question{How many years has it been since God called Abram out of his home country?}}
     the Lord appeared to Abram and said to him,
    \vspace{1\baselineskip}

    “I am God Almighty;\\
    Walk before Me, and be blameless.\\
    \vs{2}I will establish My covenant between Me and you,\\
    And I will multiply you exceedingly.”\\

    \vspace{1\baselineskip}

    \noindent\vs{3}Abram fell on his face, and God talked with him, saying,

    \vspace{1\baselineskip}

    \vs{4}“As for Me, behold, My covenant is with you,\\
    And you will be the father of a multitude of nations.\\
    \vs{5}No longer shall your name be called Abram,\\
    But your name shall be Abraham;\\
    For I have made you the father of a multitude of nations.\\

    \vspace{1\baselineskip}

    \vs{6}I will make you exceedingly fruitful, and I will make nations of you, and kings will come forth from you.
    \vs{7}I will establish My covenant between Me and you and your descendants after you throughout their generations for an everlasting covenant, to be God to you and to your descendants after you.
    \vs{8}I will give to you and to your descendants after you, the land of your sojournings, all the land of Canaan, for an everlasting possession; and I will be their God.”\\
\end{scripture}

\pagebreak


\begin{scripture}[Genesis 17:9-22]
    \vs{9}God said further to Abraham,  \marginnote{\question{How many times has Abraham had his convenant with God reaffirmed?}}
     “Now as for you, you shall keep My covenant, you and your descendants after you throughout their generations.
    \vs{10}This is My covenant, which you shall keep, between Me and you and your descendants after you: every male among you shall be circumcised.
    \vs{11}And you shall be circumcised in the flesh of your foreskin, and it shall be the sign of the covenant between Me and you.
    \vs{12}And every male among you who is eight days old shall be circumcised throughout your generations, a servant who is born in the house or who is bought with money from any foreigner, who is not of your descendants.
    \marginnote{\question{What purpose do you think this circumcision held in regards to God`s promises and Abraham`s faith in them?}}
    \vs{13}A servant who is born in your house or who is bought with your money shall surely be circumcised; thus shall My covenant be in your flesh for an everlasting covenant.
    \vs{14}But an uncircumcised male who is not circumcised in the flesh of his foreskin, that person shall be cut off from his people; he has broken My covenant.”

    \vspace{1\baselineskip}

    \vs{15}Then God said to Abraham, “As for Sarai your wife, you shall not call her name Sarai, but Sarah shall be her name. 
    \vs{16}I will bless her, and indeed I will give you a son by her. Then I will bless her, and she shall be a mother of nations; kings of peoples will come from her.”

    \vspace{1\baselineskip}

    \vs{17}Then Abraham fell on his face and laughed, \marginnote{\question{Previously Abraham had thought that Ishmael was the child that God had promised him. Did God leave any room for Abraham to misunderstand his promise this time?}}
    and said in his heart, “Will a child be born to a man one hundred years old? And will Sarah, who is ninety years old, bear a child?”
    \vs{18}And Abraham said to God, “Oh that Ishmael might live before You!”
    \vs{19}But God said, “No, but Sarah your wife will bear you a son, and you shall call his name Isaac; and I will establish My covenant with him for an everlasting covenant for his descendants after him.
    \vs{20}As for Ishmael, I have heard you; behold, I will bless him, and will make him fruitful and will multiply him exceedingly. He shall become the father of twelve princes, and I will make him a great nation.
    \vs{21}But My covenant I will establish with Isaac, whom Sarah will bear to you at this season next year.”
    \vs{22}When He finished talking with him, God went up from Abraham. 
\end{scripture}

\vspace{2\baselineskip}

\begin{scripture}[Genesis 21:1-3]
    \vs{1}Then the Lord took note of Sarah \marginnote{Note: The name which God commanded, Isaac, means "he laughs". God presumably chose this name to remind them of the laughter in the disbelief expressed by Abraham and Sarah in Genesis 17.}
    as He had said, and the Lord did for Sarah as He had promised.
    \vs{2}So Sarah conceived and bore a son to Abraham in his old age, at the appointed time of which God had spoken to him.
    \vs{3}Abraham called the name of his son who was born to him, whom Sarah bore to him, Isaac.
\end{scripture}

\pagebreak

\subsection{The Binding of Isaac}
\begin{scripture}[Genesis 22:1-19]
    \vs{1}Now it came about after these things, that God tested Abraham, \marginnote{\question{Was God tempting Abraham to do evil? Or was he testing his faithfullness? If you are unsure see James 1:13-14.}}
     and said to him, `Abraham!` And he said, `Here I am.`
    \vs{2}He said, `Take now your son, your only son, whom you love, Isaac, and go to the land of Moriah, and offer him there as a burnt offering on one of the mountains of which I will tell you.`
    \vs{3}So Abraham rose early in the morning and saddled his donkey, and took two of his young men with him and Isaac his son; and he split wood for the burnt offering, 
    and arose and went to the place of which God had told him. \marginnote{\question{What was Abraham`s motive to go and sacrafice his son?}}
    \vs{4}On the third day Abraham raised his eyes and saw the place from a distance.

    \vspace{2\baselineskip}

    \vs{5}Abraham said to his young men, 
    `Stay here with the donkey, and I and the lad will go over there; and we will worship and return to you.`
    \vs{6}Abraham took the wood of the burnt offering and laid it on Isaac his son,  \marginnote{\question{How old do you think Isaac was?}}
    and he took in his hand the fire and the knife. So the two of them walked on together.

    \vspace{2\baselineskip}

    \vs{7}Isaac spoke to Abraham his father and said, 
    `My father!` And he said, `Here I am, my son.` And he said, `Behold, the fire and the wood, but where is the lamb for the burnt offering?`
    \vs{8}Abraham said, `God will provide for Himself the lamb for the burnt offering, my son.` So the two of them walked on together. \marginnote{\question{What do you think Abraham might be thinking as they walk up this mountain?}}

    \vspace{2\baselineskip}

    \vs{9}Then they came to the place of which God had told him; and Abraham built the altar there and arranged the wood, and bound his son Isaac and laid him on the altar, on top of the wood.
    \vs{10}Abraham stretched out his hand and took the knife to slay his son.
    \vs{11}But the angel of the LORD called to him from heaven and said, `Abraham, Abraham!` And he said, `Here I am.`
    \vs{12}He said, `Do not stretch out your hand against the lad, and do nothing to him; for now I know that you fear God, since you have not withheld your son, your only son, from Me.`

    \vs{13}Then Abraham raised his \marginnote{\question{Do you think that Abraham was surprised?}}
    eyes and looked, and behold, behind [him] a ram caught in the thicket by his horns; and Abraham went and took the ram and offered him up for a burnt offering in the place of his son.
    \vs{14}Abraham called the name of that place The LORD Will Provide, as it is said to this day, `In the mount of the LORD it will be provided.`
    \vs{15}Then the angel of the LORD called to Abraham a second time from heaven,
    \vs{16}and said, `By Myself I have sworn, declares the LORD, because you have done this thing and have not withheld your son, your only son,
    \vs{17}indeed I will greatly bless you, and I will greatly multiply your seed as the stars of the heavens and as the sand which is on the seashore; and your seed shall possess the gate of their enemies.
    \vs{18}`In your seed all the nations of the earth shall be blessed, because you have obeyed My voice.`
    \vs{19}So Abraham returned to his young men, and they arose and went together to Beersheba; and Abraham lived at Beersheba.
\end{scripture}


\pagebreak
\begin{fullwidth}
    The author of Hebrews gives us further insight into Abraham`s mindset during this difficult act of faith. 
\end{fullwidth}

\vspace{1\baselineskip}

\begin{scripture}[Hebrews 11:17-19]
    \vs{17}By faith Abraham, when he was tested, \marginnote{\question{Why did Abraham have so much faith in God`s goodness?}}
    offered up Isaac, and he who had received the promises was offering up his only begotten son;
    \vs{18}it was he to whom it was said, "IN ISAAC YOUR DESCENDANTS SHALL BE CALLED."
    \vs{19}He considered that God is able to raise people even from the dead, from which he also received him back as a type.
\end{scripture}

\vspace{2\baselineskip}


\begin{fullwidth}
Now that you understand the relationship between God and Abraham, it is clear that God was not tempting Abraham to do evil, but was testing his faith in God`s promises.  
Abraham`s faith was not a blind faith, but a faith that was built on the foundation of God`s historical faithfullness to his promises and provision.
\vspace{1\baselineskip}

\noindent Do you think that this story is a fair comparison to the child sacrafice seen elsewhere in the Bible?

\vspace{5\baselineskip}
\noindent Within the conversation of Abortion, does this story have any relavence outside of the context of faith and obedience?
\end{fullwidth}

\pagebreak

\section{Did God Use Abortion as a Test of Infidelity?}
\begin{scripture}[Numbers 5:11-31]
    \vs{11}Then the LORD spoke to Moses, saying,
    \vs{12}`Speak to the sons of Israel and say to them, `If any man`s wife goes astray and is unfaithful to him,
    \vs{13}and a man has intercourse with her and it is hidden from the eyes of her husband and she is undetected, although she has defiled herself, and there is no witness against her and she has not been caught in the act,
    \vs{14}if a spirit of jealousy comes over him and he is jealous of his wife when she has defiled herself, or if a spirit of jealousy comes over him and he is jealous of his wife when she has not defiled herself,
    \vs{15}the man shall then bring his wife to the priest, and shall bring [as] an offering for her one-tenth of an ephah of barley meal; he shall not pour oil on it nor put frankincense on it, for it is a grain offering of jealousy, a grain offering of memorial, a reminder of iniquity.
    \vs{16}`Then the priest shall bring her near and have her stand before the LORD,
    \vs{17}and the priest shall take holy water in an earthenware vessel; and he shall take some of the dust that is on the floor of the tabernacle and put [it] into the water.
    \vs{18}`The priest shall then have the woman stand before the LORD and let [the hair of] the woman`s head go loose, and place the grain offering of memorial in her hands, which is the grain offering of jealousy, and in the hand of the priest is to be the water of bitterness that brings a curse.
    \vs{19}`The priest shall have her take an oath and shall say to the woman, `If no man has lain with you and if you have not gone astray into uncleanness, [being] under [the authority of] your husband, be immune to this water of bitterness that brings a curse;
    \vs{20}if you, however, have gone astray, [being] under [the authority of] your husband, and if you have defiled yourself and a man other than your husband has had intercourse with you`
    \vs{21}(then the priest shall have the woman swear with the oath of the curse, and the priest shall say to the woman), `the LORD make you a curse and an oath among your people by the LORD`S making your thigh waste away and your abdomen swell;
    \vs{22}and this water that brings a curse shall go into your stomach, and make your abdomen swell and your thigh waste away.` And the woman shall say, `Amen. Amen.`
    \vs{23}`The priest shall then write these curses on a scroll, and he shall wash them off into the water of bitterness.
    \vs{24}`Then he shall make the woman drink the water of bitterness that brings a curse, so that the water which brings a curse will go into her and [cause] bitterness.
    \vs{25}`The priest shall take the grain offering of jealousy from the woman`s hand, and he shall wave the grain offering before the LORD and bring it to the altar;
    \vs{26}and the priest shall take a handful of the grain offering as its memorial offering and offer [it] up in smoke on the altar, and afterward he shall make the woman drink the water.
    \vs{27}`When he has made her drink the water, then it shall come about, if she has defiled herself and has been unfaithful to her husband, that the water which brings a curse will go into her and [cause] bitterness, and her abdomen will swell and her thigh will waste away, and the woman will become a curse among her people.
    \vs{28}`But if the woman has not defiled herself and is clean, she will then be free and conceive children.
    \vs{29}`This is the law of jealousy: when a wife, [being] under [the authority of] her husband, goes astray and defiles herself,
    \vs{30}or when a spirit of jealousy comes over a man and he is jealous of his wife, he shall then make the woman stand before the LORD, and the priest shall apply all this law to her.
    \vs{31}`Moreover, the man will be free from guilt, but that woman shall bear her guilt.``
\end{scripture}


\pagebreak
\section{Is Masturbation Morally Equivalent to Murder?}
\begin{fullwidth}

    In discussions about abortion, sometimes people with an immature understanding of biology will draw a false equivalency between masturbation and abortion, based on the idea that during masturbation many sperms are "killed". It is a scientifically 
    agreed upon fact that sperm cells and unfertilized eggs (both called gametes) are not human beings. This is based on the fact that that only a fertilized egg (zygote) is a genetically distinct and complete organism which under the right conditions, 
    will self-assemble into a fully grown human being. \\
    While the bible does not address these circumstances exactly, there is a story that people often point to while making this false equivalency. We will read about Onan, who lived in a time when custom dictated that he was to father a child with his deceased brother`s wife. His father Judah commanded it, but he was 
    disobedient and "spilled his seed" on the ground. God killed him for what he had done.\\ The question we need to answer was what was the great sin that Onan committed? Was it the act of spilling seed, or was it the act of disobedience?
\end{fullwidth}

\vspace{2\baselineskip}

\begin{scripture}[Genesis 38:8-10]
    \vs{8}Then Judah said to Onan, `Go in to your brother`s wife, \marginnote{\question{Does Onan`s father Judah mention anything that would give you insight into the purpose of this tradition?}}
     and perform your duty as a brother-in-law to her, and raise up offspring for your brother.`
    \vs{9}Onan knew that the offspring would not be his; so when he went in to his brother`s wife, he wasted his seed on the ground in order not to give offspring to his brother.  
    \vs{10}But what he did was displeasing in the sight of the LORD; so He took his life also.\marginnote{\question{Why did Onan choose not to procreate?}}
\end{scripture}

\vspace{3\baselineskip}

\begin{fullwidth}
The story of Onan found in Genesis 38 predated the Mosaic law. Below in Deuteronomy we see the ancient near eastern custom codified into law. This custom is known as levirate marriage. 
\end{fullwidth}


\vspace{2\baselineskip}

\begin{scripture}[Deuteronomy 25:5-10]
    \vs{5}`When brothers live together and one of them dies and has no son, \marginnote{\question{What does Moses say the purpose of this tradition is?}}
    the wife of the deceased shall not be [married] outside [the family] to a strange man. Her husband`s brother shall go in to her and take her to himself as wife and perform the duty of a husband`s brother to her.
    \vs{6}`It shall be that the firstborn whom she bears shall assume the name of his dead brother, so that his name will not be blotted out from Israel. 
    \vs{7}`But if the man does not desire to take his brother`s wife, then his brother`s wife shall  \marginnote{\question{Why do you think it is important for the wife not to be married outside the family?}}
    go up to the gate to the elders and say, `My husband`s brother refuses to establish a name for his brother in Israel; he is not willing to perform the duty of a husband`s brother to me.`
    \vs{8}`Then the elders of his city shall summon him and speak to him. And [if] he persists and says, `I do not desire to take her,`
    \vs{9}then his brother`s wife shall come to him in the sight of the elders, and pull his sandal off his foot and spit in his face; and she shall declare, `Thus it is done to the man who does not build up his brother`s house.`
    \vs{10}`In Israel his name shall be called, `The house of him whose sandal is removed.`
\end{scripture}
\marginnote{\question{What was Onan`s great sin which was so displeasing to the Lord, that he took his life?}}

\vspace{2\baselineskip}

\begin{fullwidth}
It is worthwhile to note that the purpose of levirate marraige was not limited to preserving the brother`s lineage, but also to provide for the widow. In the ancient
near east, widows were vulnerable without a husband or children to support them. Levirate marriage was a way to ensure that the widow was taken care of and that 
she would stay integrated into the family. This practice also allowed for the preservation of inheritance within the extended family as the first son born from the union would
likely inherit the deceased brother`s estate. 
\end{fullwidth}
