\chapter{Murder and Killing}

\section{Unlawful Killing}

\begin{scripture}[Exodus 20:13]
    \vs{13}`You shall not murder.
\end{scripture}

\vspace{1\baselineskip}

\begin{scripture}[Exodus 21:12]
    \vs{12}\marginnote{\question{Why do you think God`s law mandate that man must die if he has taken another`s life?}}
    He who strikes a man so that he dies shall surely be put to death.
\end{scripture}

\vspace{1\baselineskip}

\begin{scripture}[Genesis 9:6]
    \vs{6}Whoever sheds man`s blood, By man his blood shall be shed, For in the image of God He made man.
\end{scripture}

\vspace{1\baselineskip}

\begin{scripture}[Numbers 35:22-25]
    \vs{22}\marginnote{\question{What crime does this most closely resemble in US law? See Appendix 1 for more information.}}
    But if he pushed him suddenly without enmity, or threw something at him without lying in wait,
    \vs{23}or with any deadly object of stone, and without seeing it dropped on him so that he died, while he was not his enemy nor seeking his injury,
    \vs{24}then the congregation shall judge between the slayer and the blood avenger according to these ordinances.
    \vs{25}The congregation shall deliver the manslayer from the hand of the blood avenger, and the congregation shall restore him to his city of refuge to which he fled; and he shall live in it until the death of the high priest who was anointed with the holy oil.
\end{scripture}

\vspace{2\baselineskip}

\begin{scripture}[Exodus 21:22-25]
    \vs{22}\marginnote{\question{What is the difference between the first and second scenario described in this passage?}}
    `If men struggle with each other and strike a woman with child so that she gives birth prematurely, yet there is no injury, he shall surely be fined as the woman`s husband may demand of him, and he shall pay as the judges [decide].
    \vs{23}`But if there is [any further] injury, then you shall appoint [as a penalty] life for life,
    \vs{24}eye for eye, tooth for tooth, hand for hand, foot for foot,
    \vs{25}burn for burn, wound for wound, bruise for bruise.
\end{scripture}
\marginnote{\question{What does this passage say about the value God places on the life of an pre-born child?}}
\vspace{3\baselineskip}

\begin{small}
Note: In 2004, the United States Congress passed the Unborn Victims of Violence Act, which recognizes a child in
utero as a legal victim if he or she is injured or killed during the commission of any of over 60 listed federal crimes of violence. Unfortunately in many
states this law has an exception for abortion. More details are provided in Appendix 1.
\end{small}

\newpage
\section{Wartime Killing}
\begin{scripture}[Deuteronomy 20:16-17]
    \vs{16}\marginnote{\question{Who commanded this action?}}
    `Only in the cities of these peoples that the LORD your God is giving you as an inheritance, you shall not leave alive anything that breathes.
    \vs{17}`But you shall utterly destroy them, the Hittite and the Amorite, the Canaanite and the Perizzite, the Hivite and the Jebusite, as the LORD your God has commanded you,
\end{scripture}
\marginnote{Note: If you are wondering why God would command such a massacre, we will be addressing this in a later chapter.}


\vspace{4\baselineskip}

\begin{scripture}[Psalm 11:5]
    \vs{5} \marginnote{\question{What do all the condemned nations have in common?}}
    The Lord tests the righteous and the wicked, And His soul hates one who loves violence.
\end{scripture}

\vspace{4\baselineskip}
\begin{scripture}[Isaiah 2:4]
    \vs{4}
    And He will judge between the nations,
    And will mediate for many peoples;
    \marginnote{\question{Isaiah`s words are a prophecy about the future reign of Jesus Christ. What will the future of war be in God`s kingdom to come?}}

    And they will beat their swords into plowshares, and their spears into pruning knives.
    Nation will not lift up a sword against nation,
    And never again will they learn war.
\end{scripture}

\vspace{4\baselineskip}
\marginnote{\question{Are these passages justification for the killing in all modern day wars? Why or why not?}}

\vspace{10\baselineskip}


\section{Killing in Self-Defense}
\begin{scripture}[Exodus 22:1-3]
    \marginnote{\question{What is the difference between the first and second scenario described in this passage?}}
    \vs{1}If a man steals an ox or a sheep and slaughters it or sells it, he shall pay five oxen for the ox and four sheep for the sheep.
    \vs{2}If the thief is caught while breaking in and is struck so that he dies, there will be no bloodguiltiness on his account.
    \vs{3}[But] if the sun has risen on him, there will be bloodguiltiness on his account. He shall surely make restitution; if he owns nothing, then he shall be sold for his theft.
\end{scripture}

\newpage
\section{Capital Punishment}
\begin{scripture}[Exodus 21:15-17]
    \vs{15}`He who strikes his father or his mother shall surely be put to death.
    \vs{16}`He who kidnaps a man, whether he sells him or he is found in his possession, shall surely be put to death.
    \vs{17}`He who curses his father or his mother shall surely be put to death.
\end{scripture}

\vspace{2\baselineskip}

\begin{scripture}[Leviticus 20:10]
    \vs{10}`If [there is] a man who commits adultery with another man`s wife, one who commits adultery with his friend`s wife, the adulterer and the adulteress shall surely be put to death.
\end{scripture}

\vspace{2\baselineskip}

\begin{scripture}[Leviticus 24:16]
    \vs{16}`Moreover, the one who blasphemes the name of the LORD shall surely be put to death; all the congregation shall certainly stone him. The alien as well as the native, when he blasphemes the Name, shall be put to death.
\end{scripture}

\vspace{2\baselineskip}

\begin{scripture}[John 19:10-11]
    \vs{10}\marginnote{\question{Does Jesus recognize the authority of the government to execute capital punishment?}}
    So Pilate said to Him, `You do not speak to me? Do You not know that I have authority to release You, and I have authority to crucify You?`
    \vs{11}Jesus answered, `You would have no authority over Me, unless it had been given you from above; for this reason he who delivered Me to you has [the] greater sin.`
\end{scripture}

\vspace{2\baselineskip}

\begin{scripture}[Romans 13:1-7]
    \vs{1}Every person is to be in subjection to the governing authorities. For there is no authority except from God, and those which exist are established by God.
    \vs{2}Therefore whoever resists authority has opposed the ordinance of God; and they who have opposed will receive condemnation upon themselves.
    \vs{3}For rulers are not a cause of fear for good behavior, but for evil. Do you want to have no fear of authority? Do what is good and you will have praise from the same;
    \vs{4}for it is a minister of God to you for good. But if you do what is evil, be afraid; for it does not bear the sword for nothing; for it is a minister of God, an avenger who brings wrath on the one who practices evil.
    \vs{5}Therefore it is necessary to be in subjection, not only because of wrath, but also for conscience` sake.
    \vs{6}For because of this you also pay taxes, for [rulers] are servants of God, devoting themselves to this very thing.
    \vs{7}Render to all what is due them: tax to whom tax [is due]; custom to whom custom; fear to whom fear; honor to whom honor.
\end{scripture}
\marginnote{\question{Need more questions on this.}}


\vspace{6\baselineskip}
