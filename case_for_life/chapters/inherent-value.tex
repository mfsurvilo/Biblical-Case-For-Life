\chapter{Inherent Value}

\section{Man`s Creation}
\marginnote{\question{What purpose did God give the animals?}}
\begin{scripture}[Gen 1:19--28]
        \vs[19]There was evening and there was morning, a fourth day.
        \vs{20}Then God said, `Let the waters teem with swarms of living creatures, and let birds fly above the earth in the open expanse of the heavens.`
        \vs{21}God created the great sea monsters and every living creature that moves, with which the waters swarmed after their kind, and every winged bird after its kind; and God saw that it was good.
        \vs{22}God blessed them, saying, `Be fruitful and multiply, and fill the waters in the seas, and let birds multiply on the earth.`
        \vs{23}There was evening and there was morning, a fifth day.
        \vs{24}Then God said, `Let the earth bring forth living creatures after their kind: cattle and creeping things and beasts of the earth after their kind`; and it was so.
        \marginnote{\question{What purpose did God give the man?}}
        \vs{25}God made the beasts of the earth after their kind, and the cattle after their kind, and everything that creeps on the ground after its kind; and God saw that it was good.
        \vs{26}Then God said, `Let Us make man in Our image, according to Our likeness; and let them rule over the fish of the sea and over the birds of the sky and over the cattle and over all the earth, and over every creeping thing that creeps on the earth.`
        \vs{27}God created man in His own image, in the image of God He created him; male and female He created them.
        \vs{28}God blessed them; and God said to them, `Be fruitful and multiply, and fill the earth, and subdue it; and rule over the fish of the sea and over the birds of the sky and over every living thing that moves on the earth.`
\end{scripture}

\marginnote{\question{In what ways was the creation of man unique from the animals?}}


\newpage
\section{Man`s Timeline}
Read the following verses and note any repeated words. 
\vspace{2\baselineskip}

\begin{scripture}[Ecclesiastes 3:11]
        \vs{11} He has made everything appropriate in its time. 
        He has also set eternity in their heart, yet so that man 
        will not find out the work which God has done from the 
        beginning even to the end.
\end{scripture}
\marginnote{\question{What timeline was man made for?}}

\vspace{2\baselineskip}

\begin{scripture}[Daniel 12:2]
    \vs{2} Many of those who sleep in the dust of the ground will awake, these to everlasting 
    life, but the others to disgrace [and] everlasting contempt.
\end{scripture}

\vspace{2\baselineskip}

\begin{scripture}[John 3:16]
    \vs{16} For God so loved the world, that He gave His
        only begotten Son, that whoever believes in Him shall not perish, but have eternal life.
\end{scripture}

\vspace{2\baselineskip}

\begin{scripture}[Romans 6:23]
        \vs{23} For the wages of sin is death, but the free gift of God is eternal life in Christ Jesus our Lord.
\end{scripture}

\vspace{2\baselineskip}

\begin{scripture}[Matthew 25:46]
        \marginnote{\question{What do we learn about God`s intention and concern for man from these verses?}}

        \vs{46} These will go away into eternal punishment, but the righteous into eternal life.
\end{scripture}

\vspace{2\baselineskip}
\marginnote{\question{What do we learn about God`s intention and concern for man from these verses?}}

\begin{scripture}[Revelation 22:5]
        \vs{5} And there will no longer be [any] night; and they will not have need of the light 
        of a lamp nor the light of the sun, because the Lord God will illumine them; and they will 
        reign forever and ever.
\end{scripture}

\vspace{2\baselineskip}

\newpage
\section{Man`s Position}
\begin{fullwidth}
In the last section we observed that man was made for eternity. God is purposeful in his creation.
The fact that he desires eternal communion with us implies that he sees our lives with tremendous inherent value. Let`s now consider the lives 
of other creatures God has made and the value He has placed on them.
\end{fullwidth}

\vspace{2\baselineskip}

\marginnote{\question{What is the punishment for taking the life of someone`s animal?}}
\begin{scripture}[Leviticus 24:17-22]
    \vs{17} If a man takes the life of any human being, he shall surely be put to death.
        \vs{18}The one who takes the life of an animal shall make it good, life for life.
        \vs{19}If a man injures his neighbor, just as he has done, so it shall be done to him:
        \marginnote{\question{What is the punishment for taking the life of another man?}}
        \vs{20}fracture for fracture, eye for eye, tooth for tooth; just as he has injured a man, so it shall be inflicted on him.
        \vs{21}Thus the one who kills an animal shall make it good, but the one who kills a man shall be put to death.
        \vs{22}There shall be one standard for you; it shall be for the stranger as well as the native, for I am the LORD your God.
        \marginnote{\question{These commands are given to the Israelites. In this context did God care if foreigners live in accordance with his laws? If they didn`t, what was the punishment?}}
\end{scripture}

\vspace{3\baselineskip}

\begin{scripture}[Gen 1:28]
    \vs{28}God blessed them; and God said to them, `Be fruitful and multiply, and fill the earth, and subdue it; and rule over the fish of the sea and over the birds of the sky and over every living thing that moves on the earth.`
\end{scripture}

\vspace{8\baselineskip}
\marginnote{\question{What do these passages reveal about the hiearchy of the value of life in God`s kingdom?}}

