\chapter{Child Sacrafice}

\begin{fullwidth}
    
Many biblical historians believe that the phrase "passing through the fire" mentioned in the upcoming is a direct reference to the act of sacraficing children
as burnt offerings to idol gods. John Day is one of the most authoritative voices on the subject. He argues that based on what is found in the Hebrew Bible, acnient Near Eastern sources and archaeological evidence 
all points to this gruesome act. In his book on the subject,Day highlights parallels between the descriptions
in the Bible and the practice of child sacrifice found in Phoenician colonies such as Carthage, where inscriptions explicitly mention sacrifice of children to the gods Baal and Molech. 

Why did people engage in child sacrafice? They believed that by offering their children to the gods they would receive blessings in return.

What does this have to do with modern day abortion? In America alone we have allowed the practice of abortion to take the lives of 25\% of the children conceived in 2020. Were these
sacraficed to the god Molech? No, they were sacraficed to the god of convenience. The god of self. The god of personal autonomy. The god of personal freedom. The god of personal choice. Reverance for God and his word 
has been replaced by reverance for self. 

If you are still struggling to see that a life inside the womb as value, this will be a difficult comparison. If however, you have come to realize God`s heart for the unborn, then you will be able to draw distinct parallels between these two abominations.
\end{fullwidth}
\footnote{John Day, \textit{Molech: A God of Human Sacrifice in the Old Testament} (Cambridge: Cambridge University Press, 1989).} 

\marginnote{Note to self maybe add visual timeline here}





\pagebreak
\subsection{Early Period: God`s Commands and Warnings}
We will look at the following passages
in chronological order to first be introduced to Molech, and then see how some of God`s own people began to be influenced their culture and engage in this detestable practice.

\vspace{1\baselineskip}

\begin{scripture}[Leviticus 18:21-24]
    \vs{21}\marginnote{Note: It is estimated that Leviticus and Deuteronomy were written between 1440 and 1400 BC.}
    You shall not give any of your offspring to offer them to Molech, nor shall you profane the name of your God; I am the LORD.
    \vs{22}You shall not sleep with a male as one sleeps with a female; it is an abomination.
    \vs{23}Also you shall not have sexual intercourse with any animal to be defiled with it, nor shall any woman stand before an animal to mate with it; it is a perversion.
    \vs{24}Do not defile yourselves by any of these things; for by all these things the nations which I am driving out from you have become defiled.
\end{scripture}

\vspace{1\baselineskip}

\begin{scripture}[Leviticus 20:1-5]
    \vs{1}Then the LORD spoke to Moses, saying, \marginnote{\question{What specific consequences does God prescribe for those who sacrafice their children to Molech?}}
    \vs{2}`You shall also say to the sons of Israel: `Any man from the sons of Israel or from the aliens sojourning in Israel who gives any of his offspring to Molech, 
    shall surely be put to death; the people of the land shall stone him with stones.
    \vs{3}`I will also set My face against that man and will cut him off from among his people, because he has given some of his offspring to Molech, so as to defile My sanctuary and to profane My holy name.
    \vs{4}`If the people of the land, however, should ever disregard that man when he gives any of his offspring to Molech, so as not to put him to death,\marginnote{\question{Some of God`s people were not willing to hand down the judgement that God commanded. What was the specific punishment for those who turned a blind eye?}}
    \vs{5}then I Myself will set My face against that man and against his family, and I will cut off from among their people both him and all those who play the harlot after him, by playing the harlot after Molech.
\end{scripture}
\vspace{1\baselineskip}

\begin{scripture}[Deuteronomy 2:19]
    \vs{19}`When you come opposite the sons of Ammon, do not harass them nor provoke them, for I will not give you any of the land of the sons of Ammon as a possession, because I have given it to the sons of Lot as a possession.`
\end{scripture}

\marginnote{\question{For which man is the punishment more severe? Why might this be?}}


\vspace{1\baselineskip}

\begin{scripture}[Deuteronomy 12:29-32]
    \vs{29}“When the Lord your God cuts off before you the nations which you are going in to dispossess, and you dispossess them and dwell in their land,
    \vs{30}beware that you are not ensnared to follow them, after they are destroyed before you,     
    and that you do not inquire after their gods, saying, `How do these nations serve their gods, that I also may do likewise?`
    \vs{31}You shall not behave thus toward the Lord your God, for every abominable act which the Lord hates they have done for their gods; for they even burn their sons and daughters in the fire to their gods.
    \vs{32}Whatever I command you, you shall be careful to do; you shall not add to nor take away from it. \marginnote{\question{God gives this warning because He knows that his people will be tempted to follow the practices of the nations they are conquering.
    Culture tells believers that in order to be loving we must be tolerant. Can you be tolerant of evil and follow God? Do you need to change your perspective on what it means to be loving?}}
\end{scripture}


\vspace{1\baselineskip}


\begin{scripture}[Deuteronomy 18:9-10]
    \vs{9}When you enter the land which the LORD your God gives you, you shall not learn to imitate the detestable things of those nations.
    \vs{10}There shall not be found among you anyone who makes his son or his daughter pass through the fire, one who uses divination, one who practices witchcraft, or one who interprets omens, or a sorcerer,
\end{scripture}
















\pagebreak
\subsection{Later Period: Disobedience of God`s People}
\begin{fullwidth}
The period of the Judges and the Kings was marked by a cycle of disobedience, judgement, repentance by God`s people. In their disobedience, the nation that was to be set apart for God`s purposes
began to adopt the practices of the nations around them. Peter writes in his first letter, that the followers of Christ are now a chosen people, set apart for His purposes. We are to be holy as He is holy.
The context of the following passages may be different than our modern circumstances, but the principles remain the same. God`s people are to be set apart for His purposes and ought to live in obedience to His commands.
\end{fullwidth}

\vspace{2\baselineskip}

\begin{scripture}[Judges 2:11-13]
    \vs{11}Then \marginnote{Note: It is estimated that Judges was written regarding the time period of 1380-1050BC.}
    the sons of Israel did evil in the sight of the LORD and served the Baals,
    \vs{12}and they abandoned the LORD, the God of their fathers, who had brought them out of the land of Egypt; and they followed other gods from the gods of the peoples who were around them, and bowed down to them; so they provoked the LORD to anger.
    \vs{13}They abandoned the LORD and served Baal and the Ashtaroth.
\end{scripture}

\vspace{2\baselineskip}

\begin{scripture}[1 Kings 11:4-8]
    \vs{4}For when Solomon was old, his wives turned his heart away after other gods; and his heart was not wholly devoted to the LORD his God, as the heart of David his father [had been].
    \vs{5}For Solomon went after Ashtoreth the goddess of the Sidonians and after Milcom the detestable idol of the Ammonites.
    \vs{6}Solomon did what was evil in the sight of the LORD, and did not follow the LORD fully, as David his father [had done].
    \vs{7}Then Solomon built a high place for Chemosh the detestable idol of Moab, on the mountain which is east of Jerusalem, and for Molech the detestable idol of the sons of Ammon.
    \vs{8}Thus also he did for all his foreign wives, who burned incense and sacrificed to their gods.
\end{scripture}

\vspace{2\baselineskip}

\begin{scripture}[2 Chronicles 28:1-4]
    \vs{1}Ahaz [was] twenty years old when he became king, and he reigned sixteen years in Jerusalem; and he did not do right in the sight of the LORD as David his father [had done].
    \vs{2}But he walked in the ways of the kings of Israel; he also made molten images for the Baals.
    \vs{3}Moreover, he burned incense in the valley of Ben-hinnom and burned his sons in fire, according to the abominations of the nations whom the LORD had driven out before the sons of Israel.
    \vs{4}He sacrificed and burned incense on the high places, on the hills and under every green tree.
\end{scripture}

\vspace{2\baselineskip}

\begin{scripture}[2 Kings 21:1-6]
    \vs{1}Manasseh was twelve years old when he became king, and he reigned fifty-five years in Jerusalem; and his mother`s name was Hephzibah.
    \vs{2}He did evil in the sight of the LORD, according to the abominations of the nations whom the LORD dispossessed before the sons of Israel.
    \vs{3}For he rebuilt the high places which Hezekiah his father had destroyed; and he erected altars for Baal and made an Asherah, as Ahab king of Israel had done, and worshiped all the host of heaven and served them.
    \vs{4}He built altars in the house of the LORD, of which the LORD had said, `In Jerusalem I will put My name.`
    \vs{5}For he built altars for all the host of heaven in the two courts of the house of the LORD.
    \vs{6}He made his son pass through the fire, practiced witchcraft and used divination, and dealt with mediums and spiritists. He did much evil in the sight of the LORD provoking [Him to anger].
    \marginnote{\question{Is it possible that we as a nation are provoking the Lord`s righteous wrath? If so, what might that look like?}}
\end{scripture}

\marginnote{\question{What does it mean that some nations were dispossessed? Why was this their fate?}}

\vspace{2\baselineskip}

\begin{scripture}[2 Kings 21:11-12]
    \vs{11}`Because Manasseh king of Judah has done these abominations, having done wickedly more than all the Amorites did who [were] before him, and has also made Judah sin with his idols;
    \vs{12}therefore thus says the LORD, the God of Israel, `Behold, I am bringing [such] calamity on Jerusalem and Judah, that whoever hears of it, both his ears will tingle.
\end{scripture}

\vspace{2\baselineskip}

\begin{scripture}[2 Kings 21:16]
    \vs{16}Moreover, Manasseh shed very much innocent blood until he had filled Jerusalem from one end to another; besides his sin with which he made Judah sin, in doing evil in the sight of the LORD.
\end{scripture}

\vspace{2\baselineskip}

\pagebreak
\subsection{Consequences of Straying: Poetic and Prophetic Rebuke}

\begin{scripture}[Jeremiah 7:30-31]
    \vs{30}For the sons of Judah have done that which is evil in My sight,” declares the Lord, “they have set their detestable things in the house which is called by My name, to defile it.
    \vs{31}They have built the high places of Topheth, which is in the Valley of Ben-hinnom, to burn their sons and their daughters in the fire, which I did not command, and it did not come into My mind.
\end{scripture}

\vspace{2\baselineskip}

\begin{scripture}[Psalm 106:37-38]
    \vs{37}They even sacrificed their sons and their daughters to the demons,
    \vs{38}And shed innocent blood,
    The blood of their sons and their daughters
    Whom they sacrificed to the idols of Canaan;
    And the land was defiled with the blood.  \marginnote{\question{What might it mean that the land was defiled?}}
\end{scripture}

\vspace{2\baselineskip}

\begin{scripture}[Hosea4:1-3]
    \vs{1}Listen to the word of the Lord, you sons of Israel,
    Because the Lord has a case against the inhabitants of the land,
    For there is no faithfulness, nor loyalty,
    Nor knowledge of God in the land.
    \vs{2}There is oath-taking, denial, murder, stealing, and adultery.
    They employ violence, so that bloodshed follows bloodshed.
    \vs{3}Therefore the land mourns,
    And everyone who lives in it languishes
    Along with the animals of the field and the birds of the sky,
    And even the fish of the sea disappear.
\end{scripture}

\vspace{2\baselineskip}

\begin{scripture}[Amos 1:13-15]
    \vs{13}Thus says the LORD, `For three transgressions of the sons of Ammon and for four I will not revoke its [punishment], Because they ripped open the pregnant women of Gilead In order to enlarge their borders.
    \vs{14}`So I will kindle a fire on the wall of Rabbah And it will consume her citadels Amid war cries on the day of battle, And a storm on the day of tempest.
    \vs{15}`Their king will go into exile, He and his princes together,` says the LORD. 
    \marginnote{\question{Was God watching the actions of the Ammonites?  What did he promise to do to them?}}
\end{scripture}